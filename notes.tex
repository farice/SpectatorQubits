\documentclass[12]{article}

\usepackage{amssymb,amsmath}

%\usepackage{refcheck}
\usepackage{subcaption}
\usepackage{graphicx}
\usepackage{amssymb}
\usepackage{mathrsfs}
\usepackage{amsmath}
\usepackage{latexsym}
\usepackage{amssymb}
\usepackage{enumerate}
\usepackage{fullpage} 
\usepackage{setspace}
\usepackage{color}
%\usepackage{ dsfont }
\usepackage{float}
\usepackage{physics}
\usepackage{hyperref}

\hypersetup{colorlinks=true,linkcolor=blue, linktocpage}

%new math symbols taking no arguments
\newcommand\0{\mathbf{0}}
\newcommand\CC{\mathbb{C}}
\newcommand\FF{\mathbb{F}}
\newcommand\NN{\mathbb{N}}
\newcommand\QQ{\mathbb{Q}}
\newcommand\RR{\mathbb{R}}
\newcommand\ZZ{\mathbb{Z}}
\newcommand\bb{\mathbf{b}}
\newcommand\kk{\Bbbk}
\newcommand\mm{\mathfrak{m}}
\newcommand\pp{\mathfrak{p}}
\newcommand\xx{\mathbf{x}}
\newcommand\yy{\mathbf{y}}
\newcommand\GL{\mathit{GL}}
\newcommand\into{\hookrightarrow}
\newcommand\nsub{\trianglelefteq}
\newcommand\onto{\twoheadrightarrow}
\newcommand\minus{\smallsetminus}
\newcommand\goesto{\rightsquigarrow}
\newcommand\nsubneq{\vartriangleleft}

%redefined math symbols taking no arguments
\newcommand\<{\langle}
\renewcommand\>{\rangle}
\renewcommand\iff{\Leftrightarrow}
\renewcommand\phi{\varphi}
\renewcommand\implies{\Rightarrow}

%new math symbols taking arguments
\newcommand\ol[1]{{\overline{#1}}}

%redefined math symbols taking arguments
\renewcommand\mod[1]{\ (\mathrm{mod}\ #1)}

%roman font math operators
\DeclareMathOperator\aut{Aut}

%for easy 2 x 2 matrices
\newcommand\twobytwo[1]{\left[\begin{array}{@{}cc@{}}#1\end{array}\right]}

%for easy column vectors of size 2
\newcommand\tworow[1]{\left[\begin{array}{@{}c@{}}#1\end{array}\right]}

\newtheorem{theorem}{Theorem}[section]
\newtheorem{corollary}{Corollary}[theorem]
\newtheorem{lemma}[theorem]{Lemma}
\newtheorem{exercise}[theorem]{Exercise}
\newtheorem{definition}[theorem]{Definition}

\title{Spectator Qubit Notes}
\author{Faris Sbahi}

\begin{document}
\maketitle

\section{Gate Set Tomography}

\subsection{Notation}

Let $\mathcal{H}$ be a Hilbert space of dimension $d$. Hence, we have $d \times d$ density operators $\rho$ s.t. $\rho \succeq 0$ and $\Tr \rho = 1$. We are interested in characterizing an unknown process $G$ which acts on these density operators. Because $G$ is then a CPTP map, we can use Kraus's form

\begin{align*}
G[\rho] &= \sum_i A_i \rho A_i^\dag 	
\end{align*}

where $\{A_i\}$ satisfy $\sum_i A_i A_i^\dag = I$.

In tomography, it is useful to represent quantum processes using the Hilbert-Schmidt space of matrices on $\mathcal{H}$, denoted $B(\mathcal{H})$, in which any $d \times d$ matrix $X$ is a vector $\ket{X}$ (or $\bra{X}$) of size $d^2$.

So, let $\{E_i\}$ be a POVM and then

\begin{align*}
\Pr(k) &= \bra{E_k}\ket{\rho}
\end{align*}

if we choose $\Tr[X^\dag Y] \equiv \bra{X}\ket{Y}$ as the inner product on $B(\mathcal{H})$.

Furthermore, because $G$ is linear it can be represented as a $d^2 \times d^2$ matrix. Hence, if we perform the measurement given by $\{E_i\}$ after process $G$ has transformed $\rho$, we have

\begin{align*}
\Pr(k) &= \bra{E_k}G\ket{\rho}
\end{align*}

\subsection{Tomography}

Assume that $\ket{\rho}$ is unknown.

\subsection{Gate Set Tomography}

\end{document}
